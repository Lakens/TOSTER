% Options for packages loaded elsewhere
\PassOptionsToPackage{unicode}{hyperref}
\PassOptionsToPackage{hyphens}{url}
\PassOptionsToPackage{dvipsnames,svgnames,x11names}{xcolor}
%
\documentclass[
]{interact}

\usepackage{amsmath,amssymb}
\usepackage{iftex}
\ifPDFTeX
  \usepackage[T1]{fontenc}
  \usepackage[utf8]{inputenc}
  \usepackage{textcomp} % provide euro and other symbols
\else % if luatex or xetex
  \usepackage{unicode-math}
  \defaultfontfeatures{Scale=MatchLowercase}
  \defaultfontfeatures[\rmfamily]{Ligatures=TeX,Scale=1}
\fi
\usepackage{lmodern}
\ifPDFTeX\else  
    % xetex/luatex font selection
\fi
% Use upquote if available, for straight quotes in verbatim environments
\IfFileExists{upquote.sty}{\usepackage{upquote}}{}
\IfFileExists{microtype.sty}{% use microtype if available
  \usepackage[]{microtype}
  \UseMicrotypeSet[protrusion]{basicmath} % disable protrusion for tt fonts
}{}
\makeatletter
\@ifundefined{KOMAClassName}{% if non-KOMA class
  \IfFileExists{parskip.sty}{%
    \usepackage{parskip}
  }{% else
    \setlength{\parindent}{0pt}
    \setlength{\parskip}{6pt plus 2pt minus 1pt}}
}{% if KOMA class
  \KOMAoptions{parskip=half}}
\makeatother
\usepackage{xcolor}
\usepackage{soul}
\setlength{\emergencystretch}{3em} % prevent overfull lines
\setcounter{secnumdepth}{5}
% Make \paragraph and \subparagraph free-standing
\ifx\paragraph\undefined\else
  \let\oldparagraph\paragraph
  \renewcommand{\paragraph}[1]{\oldparagraph{#1}\mbox{}}
\fi
\ifx\subparagraph\undefined\else
  \let\oldsubparagraph\subparagraph
  \renewcommand{\subparagraph}[1]{\oldsubparagraph{#1}\mbox{}}
\fi

\usepackage{color}
\usepackage{fancyvrb}
\newcommand{\VerbBar}{|}
\newcommand{\VERB}{\Verb[commandchars=\\\{\}]}
\DefineVerbatimEnvironment{Highlighting}{Verbatim}{commandchars=\\\{\}}
% Add ',fontsize=\small' for more characters per line
\usepackage{framed}
\definecolor{shadecolor}{RGB}{241,243,245}
\newenvironment{Shaded}{\begin{snugshade}}{\end{snugshade}}
\newcommand{\AlertTok}[1]{\textcolor[rgb]{0.68,0.00,0.00}{#1}}
\newcommand{\AnnotationTok}[1]{\textcolor[rgb]{0.37,0.37,0.37}{#1}}
\newcommand{\AttributeTok}[1]{\textcolor[rgb]{0.40,0.45,0.13}{#1}}
\newcommand{\BaseNTok}[1]{\textcolor[rgb]{0.68,0.00,0.00}{#1}}
\newcommand{\BuiltInTok}[1]{\textcolor[rgb]{0.00,0.23,0.31}{#1}}
\newcommand{\CharTok}[1]{\textcolor[rgb]{0.13,0.47,0.30}{#1}}
\newcommand{\CommentTok}[1]{\textcolor[rgb]{0.37,0.37,0.37}{#1}}
\newcommand{\CommentVarTok}[1]{\textcolor[rgb]{0.37,0.37,0.37}{\textit{#1}}}
\newcommand{\ConstantTok}[1]{\textcolor[rgb]{0.56,0.35,0.01}{#1}}
\newcommand{\ControlFlowTok}[1]{\textcolor[rgb]{0.00,0.23,0.31}{#1}}
\newcommand{\DataTypeTok}[1]{\textcolor[rgb]{0.68,0.00,0.00}{#1}}
\newcommand{\DecValTok}[1]{\textcolor[rgb]{0.68,0.00,0.00}{#1}}
\newcommand{\DocumentationTok}[1]{\textcolor[rgb]{0.37,0.37,0.37}{\textit{#1}}}
\newcommand{\ErrorTok}[1]{\textcolor[rgb]{0.68,0.00,0.00}{#1}}
\newcommand{\ExtensionTok}[1]{\textcolor[rgb]{0.00,0.23,0.31}{#1}}
\newcommand{\FloatTok}[1]{\textcolor[rgb]{0.68,0.00,0.00}{#1}}
\newcommand{\FunctionTok}[1]{\textcolor[rgb]{0.28,0.35,0.67}{#1}}
\newcommand{\ImportTok}[1]{\textcolor[rgb]{0.00,0.46,0.62}{#1}}
\newcommand{\InformationTok}[1]{\textcolor[rgb]{0.37,0.37,0.37}{#1}}
\newcommand{\KeywordTok}[1]{\textcolor[rgb]{0.00,0.23,0.31}{#1}}
\newcommand{\NormalTok}[1]{\textcolor[rgb]{0.00,0.23,0.31}{#1}}
\newcommand{\OperatorTok}[1]{\textcolor[rgb]{0.37,0.37,0.37}{#1}}
\newcommand{\OtherTok}[1]{\textcolor[rgb]{0.00,0.23,0.31}{#1}}
\newcommand{\PreprocessorTok}[1]{\textcolor[rgb]{0.68,0.00,0.00}{#1}}
\newcommand{\RegionMarkerTok}[1]{\textcolor[rgb]{0.00,0.23,0.31}{#1}}
\newcommand{\SpecialCharTok}[1]{\textcolor[rgb]{0.37,0.37,0.37}{#1}}
\newcommand{\SpecialStringTok}[1]{\textcolor[rgb]{0.13,0.47,0.30}{#1}}
\newcommand{\StringTok}[1]{\textcolor[rgb]{0.13,0.47,0.30}{#1}}
\newcommand{\VariableTok}[1]{\textcolor[rgb]{0.07,0.07,0.07}{#1}}
\newcommand{\VerbatimStringTok}[1]{\textcolor[rgb]{0.13,0.47,0.30}{#1}}
\newcommand{\WarningTok}[1]{\textcolor[rgb]{0.37,0.37,0.37}{\textit{#1}}}

\providecommand{\tightlist}{%
  \setlength{\itemsep}{0pt}\setlength{\parskip}{0pt}}\usepackage{longtable,booktabs,array}
\usepackage{calc} % for calculating minipage widths
% Correct order of tables after \paragraph or \subparagraph
\usepackage{etoolbox}
\makeatletter
\patchcmd\longtable{\par}{\if@noskipsec\mbox{}\fi\par}{}{}
\makeatother
% Allow footnotes in longtable head/foot
\IfFileExists{footnotehyper.sty}{\usepackage{footnotehyper}}{\usepackage{footnote}}
\makesavenoteenv{longtable}
\usepackage{graphicx}
\makeatletter
\def\maxwidth{\ifdim\Gin@nat@width>\linewidth\linewidth\else\Gin@nat@width\fi}
\def\maxheight{\ifdim\Gin@nat@height>\textheight\textheight\else\Gin@nat@height\fi}
\makeatother
% Scale images if necessary, so that they will not overflow the page
% margins by default, and it is still possible to overwrite the defaults
% using explicit options in \includegraphics[width, height, ...]{}
\setkeys{Gin}{width=\maxwidth,height=\maxheight,keepaspectratio}
% Set default figure placement to htbp
\makeatletter
\def\fps@figure{htbp}
\makeatother
\newlength{\cslhangindent}
\setlength{\cslhangindent}{1.5em}
\newlength{\csllabelwidth}
\setlength{\csllabelwidth}{3em}
\newlength{\cslentryspacingunit} % times entry-spacing
\setlength{\cslentryspacingunit}{\parskip}
\newenvironment{CSLReferences}[2] % #1 hanging-ident, #2 entry spacing
 {% don't indent paragraphs
  \setlength{\parindent}{0pt}
  % turn on hanging indent if param 1 is 1
  \ifodd #1
  \let\oldpar\par
  \def\par{\hangindent=\cslhangindent\oldpar}
  \fi
  % set entry spacing
  \setlength{\parskip}{#2\cslentryspacingunit}
 }%
 {}
\usepackage{calc}
\newcommand{\CSLBlock}[1]{#1\hfill\break}
\newcommand{\CSLLeftMargin}[1]{\parbox[t]{\csllabelwidth}{#1}}
\newcommand{\CSLRightInline}[1]{\parbox[t]{\linewidth - \csllabelwidth}{#1}\break}
\newcommand{\CSLIndent}[1]{\hspace{\cslhangindent}#1}

\usepackage{booktabs}
\usepackage{longtable}
\usepackage{array}
\usepackage{multirow}
\usepackage{wrapfig}
\usepackage{float}
\usepackage{colortbl}
\usepackage{pdflscape}
\usepackage{tabu}
\usepackage{threeparttable}
\usepackage{threeparttablex}
\usepackage[normalem]{ulem}
\usepackage{makecell}
\usepackage{xcolor}
\usepackage{orcidlink}
\makeatletter
\makeatother
\makeatletter
\makeatother
\makeatletter
\@ifpackageloaded{caption}{}{\usepackage{caption}}
\AtBeginDocument{%
\ifdefined\contentsname
  \renewcommand*\contentsname{Table of contents}
\else
  \newcommand\contentsname{Table of contents}
\fi
\ifdefined\listfigurename
  \renewcommand*\listfigurename{List of Figures}
\else
  \newcommand\listfigurename{List of Figures}
\fi
\ifdefined\listtablename
  \renewcommand*\listtablename{List of Tables}
\else
  \newcommand\listtablename{List of Tables}
\fi
\ifdefined\figurename
  \renewcommand*\figurename{Figure}
\else
  \newcommand\figurename{Figure}
\fi
\ifdefined\tablename
  \renewcommand*\tablename{Table}
\else
  \newcommand\tablename{Table}
\fi
}
\@ifpackageloaded{float}{}{\usepackage{float}}
\floatstyle{ruled}
\@ifundefined{c@chapter}{\newfloat{codelisting}{h}{lop}}{\newfloat{codelisting}{h}{lop}[chapter]}
\floatname{codelisting}{Listing}
\newcommand*\listoflistings{\listof{codelisting}{List of Listings}}
\makeatother
\makeatletter
\@ifpackageloaded{caption}{}{\usepackage{caption}}
\@ifpackageloaded{subcaption}{}{\usepackage{subcaption}}
\makeatother
\makeatletter
\@ifpackageloaded{tcolorbox}{}{\usepackage[skins,breakable]{tcolorbox}}
\makeatother
\makeatletter
\@ifundefined{shadecolor}{\definecolor{shadecolor}{rgb}{.97, .97, .97}}
\makeatother
\makeatletter
\makeatother
\makeatletter
\makeatother
\ifLuaTeX
  \usepackage{selnolig}  % disable illegal ligatures
\fi
\IfFileExists{bookmark.sty}{\usepackage{bookmark}}{\usepackage{hyperref}}
\IfFileExists{xurl.sty}{\usepackage{xurl}}{} % add URL line breaks if available
\urlstyle{same} % disable monospaced font for URLs
\hypersetup{
  pdftitle={Demo Taylor and Francis template},
  pdfauthor={Michael J Mahoney; Another One; Someone Else},
  pdfkeywords={template, demo},
  colorlinks=true,
  linkcolor={blue},
  filecolor={Maroon},
  citecolor={Blue},
  urlcolor={Blue},
  pdfcreator={LaTeX via pandoc}}

\title{Demo Taylor and Francis template}
\author{Michael J
Mahoney$\textsuperscript{1}$~\orcidlink{0000-0003-2402-304X}, Another
One$\textsuperscript{1}$, Someone Else$\textsuperscript{2}$}

\thanks{CONTACT: Michael J
Mahoney. Email: \href{mailto:fake.email@fakeyfake.com}{\nolinkurl{fake.email@fakeyfake.com}}. Someone
Else. Email: \href{mailto:test@email.com}{\nolinkurl{test@email.com}}. }
\begin{document}
\captionsetup{labelsep=space}
\maketitle
\textsuperscript{1} Graduate Program in Environmental Science, State
University of New York College of Environmental Science and
Forestry, Syracuse, NY, USA\\ \textsuperscript{2} Department of
Sustainable Resources Management, State University of New York College
of Environmental Science and Forestry, Syracuse, NY, USA
\begin{abstract}
This document is only a demo explaining how to use the template.
\end{abstract}
\begin{keywords}
\def\sep{;\ }
template\sep 
demo
\end{keywords}
\ifdefined\Shaded\renewenvironment{Shaded}{\begin{tcolorbox}[interior hidden, sharp corners, boxrule=0pt, breakable, borderline west={3pt}{0pt}{shadecolor}, frame hidden, enhanced]}{\end{tcolorbox}}\fi

\hypertarget{sec-intro}{%
\section{Introduction}\label{sec-intro}}

This is an example of how to use this template to render journal
articles. This template is inspired by the Taylor and Francis rticles
template for rmarkdown, repurposed for the Quarto publishing system.

This quarto extension format supports PDF and HTML outputs. This
template is primarily focused on generating acceptable {\LaTeX} outputs
from Quarto, but renders an acceptable HTML output using the standard
Quarto options.

\hypertarget{quarto}{%
\section{Quarto}\label{quarto}}

Quarto enables you to weave together content and executable code into a
finished document. To learn more about Quarto see
\url{https://quarto.org}.

\hypertarget{running-code}{%
\section{Running Code}\label{running-code}}

When you click the \textbf{Render} button a document will be generated
that includes both content and the output of embedded code. You can
embed code like this:

\begin{verbatim}
[1] 2
\end{verbatim}

This format hide chunks by default, but you can set \texttt{echo} option
to \texttt{true} locally in the chunk:

\begin{Shaded}
\begin{Highlighting}[]
\CommentTok{\# install.packages("broom")}
\CommentTok{\# install.packages("kableExtra")}
\FunctionTok{data}\NormalTok{(}\StringTok{"quine"}\NormalTok{, }\AttributeTok{package =} \StringTok{"MASS"}\NormalTok{)}
\NormalTok{m\_pois }\OtherTok{\textless{}{-}} \FunctionTok{glm}\NormalTok{(Days }\SpecialCharTok{\textasciitilde{}}\NormalTok{ (Eth }\SpecialCharTok{+}\NormalTok{ Sex }\SpecialCharTok{+}\NormalTok{ Age }\SpecialCharTok{+}\NormalTok{ Lrn)}\SpecialCharTok{\^{}}\DecValTok{2}\NormalTok{, }\AttributeTok{data =}\NormalTok{ quine, }\AttributeTok{family =}\NormalTok{ poisson)}
\NormalTok{kableExtra}\SpecialCharTok{::}\FunctionTok{kable\_styling}\NormalTok{(}
\NormalTok{  kableExtra}\SpecialCharTok{::}\FunctionTok{kbl}\NormalTok{(broom}\SpecialCharTok{::}\FunctionTok{tidy}\NormalTok{(m\_pois))}
\NormalTok{)}
\end{Highlighting}
\end{Shaded}

\hypertarget{tbl-glm}{}
\begin{table}
\caption{\label{tbl-glm}A table. }\tabularnewline

\centering
\begin{tabular}[t]{l|r|r|r|r}
\hline
term & estimate & std.error & statistic & p.value\\
\hline
(Intercept) & 2.9324591 & 0.0982638 & 29.8427305 & 0.0000000\\
\hline
EthN & -0.1739938 & 0.1213351 & -1.4339937 & 0.1515741\\
\hline
SexM & -0.7145197 & 0.1222943 & -5.8426235 & 0.0000000\\
\hline
AgeF1 & -0.0426993 & 0.1269111 & -0.3364507 & 0.7365310\\
\hline
AgeF2 & -0.0863239 & 0.1616403 & -0.5340495 & 0.5933073\\
\hline
AgeF3 & -0.1528978 & 0.1189753 & -1.2851227 & 0.1987494\\
\hline
LrnSL & 0.2160818 & 0.1455811 & 1.4842716 & 0.1377369\\
\hline
EthN:SexM & 0.4390243 & 0.0920790 & 4.7679077 & 0.0000019\\
\hline
EthN:AgeF1 & -0.9288934 & 0.1465738 & -6.3373786 & 0.0000000\\
\hline
EthN:AgeF2 & -1.3339773 & 0.1350383 & -9.8785113 & 0.0000000\\
\hline
EthN:AgeF3 & -0.1124246 & 0.1347842 & -0.8341080 & 0.4042202\\
\hline
EthN:LrnSL & 0.2641524 & 0.1137843 & 2.3215200 & 0.0202588\\
\hline
SexM:AgeF1 & -0.0556536 & 0.1630311 & -0.3413682 & 0.7328264\\
\hline
SexM:AgeF2 & 1.0994244 & 0.1528125 & 7.1945973 & 0.0000000\\
\hline
SexM:AgeF3 & 1.1594892 & 0.1385899 & 8.3663319 & 0.0000000\\
\hline
SexM:LrnSL & 0.0414270 & 0.1371756 & 0.3019998 & 0.7626522\\
\hline
AgeF1:LrnSL & -0.1301879 & 0.1568800 & -0.8298561 & 0.4066201\\
\hline
AgeF2:LrnSL & 0.3734020 & 0.1456293 & 2.5640585 & 0.0103456\\
\hline
AgeF3:LrnSL & NA & NA & NA & NA\\
\hline
\end{tabular}
\end{table}

\hypertarget{markdown-basics}{%
\section{Markdown Basics}\label{markdown-basics}}

This section of the template is adapted from
\href{https://quarto.org/docs/authoring/markdown-basics.html}{Quarto's
documentation on Markdown basics}.

\hypertarget{text-formatting}{%
\subsection{Text Formatting}\label{text-formatting}}

\begin{longtable}[]{@{}
  >{\raggedright\arraybackslash}p{(\columnwidth - 2\tabcolsep) * \real{0.5000}}
  >{\raggedright\arraybackslash}p{(\columnwidth - 2\tabcolsep) * \real{0.4444}}@{}}
\toprule\noalign{}
\begin{minipage}[b]{\linewidth}\raggedright
Markdown Syntax
\end{minipage} & \begin{minipage}[b]{\linewidth}\raggedright
Output
\end{minipage} \\
\midrule\noalign{}
\endhead
\bottomrule\noalign{}
\endlastfoot
\begin{minipage}[t]{\linewidth}\raggedright
\begin{verbatim}
*italics* and **bold**
\end{verbatim}
\end{minipage} & \emph{italics} and \textbf{bold} \\
\begin{minipage}[t]{\linewidth}\raggedright
\begin{verbatim}
superscript^2^ / subscript~2~
\end{verbatim}
\end{minipage} & superscript\textsuperscript{2} /
subscript\textsubscript{2} \\
\begin{minipage}[t]{\linewidth}\raggedright
\begin{verbatim}
~~strikethrough~~
\end{verbatim}
\end{minipage} & \st{strikethrough} \\
\begin{minipage}[t]{\linewidth}\raggedright
\begin{verbatim}
`verbatim code`
\end{verbatim}
\end{minipage} & \texttt{verbatim\ code} \\
\end{longtable}

\hypertarget{headings}{%
\subsection{Headings}\label{headings}}

\begin{longtable}[]{@{}
  >{\raggedright\arraybackslash}p{(\columnwidth - 2\tabcolsep) * \real{0.3056}}
  >{\raggedright\arraybackslash}p{(\columnwidth - 2\tabcolsep) * \real{0.5000}}@{}}
\toprule\noalign{}
\begin{minipage}[b]{\linewidth}\raggedright
Markdown Syntax
\end{minipage} & \begin{minipage}[b]{\linewidth}\raggedright
Output
\end{minipage} \\
\midrule\noalign{}
\endhead
\bottomrule\noalign{}
\endlastfoot
\begin{minipage}[t]{\linewidth}\raggedright
\begin{verbatim}
# Header 1
\end{verbatim}
\end{minipage} & \begin{minipage}[t]{\linewidth}\raggedright
\hypertarget{header-1}{%
\section{Header 1}\label{header-1}}
\end{minipage} \\
\begin{minipage}[t]{\linewidth}\raggedright
\begin{verbatim}
## Header 2
\end{verbatim}
\end{minipage} & \begin{minipage}[t]{\linewidth}\raggedright
\hypertarget{header-2}{%
\subsection{Header 2}\label{header-2}}
\end{minipage} \\
\begin{minipage}[t]{\linewidth}\raggedright
\begin{verbatim}
### Header 3
\end{verbatim}
\end{minipage} & \begin{minipage}[t]{\linewidth}\raggedright
\hypertarget{header-3}{%
\subsubsection{Header 3}\label{header-3}}
\end{minipage} \\
\end{longtable}

\hypertarget{equations}{%
\subsection{Equations}\label{equations}}

Use \texttt{\$} delimiters for inline math and \texttt{\$\$} delimiters
for display math. For example:

\begin{longtable}[]{@{}
  >{\raggedright\arraybackslash}p{(\columnwidth - 2\tabcolsep) * \real{0.4444}}
  >{\raggedright\arraybackslash}p{(\columnwidth - 2\tabcolsep) * \real{0.3611}}@{}}
\toprule\noalign{}
\begin{minipage}[b]{\linewidth}\raggedright
Markdown Syntax
\end{minipage} & \begin{minipage}[b]{\linewidth}\raggedright
Output
\end{minipage} \\
\midrule\noalign{}
\endhead
\bottomrule\noalign{}
\endlastfoot
\begin{minipage}[t]{\linewidth}\raggedright
\begin{verbatim}
inline math: $E = mc^{2}$
\end{verbatim}
\end{minipage} & inline math: \(E=mc^{2}\) \\
\begin{minipage}[t]{\linewidth}\raggedright
\begin{verbatim}
display math:

$$E = mc^{2}$$
\end{verbatim}
\end{minipage} & \begin{minipage}[t]{\linewidth}\raggedright
display math:\\
\[E = mc^{2}\]\strut
\end{minipage} \\
\end{longtable}

If assigned an ID, display math equations will be automatically
numbered:

\begin{equation}\protect\hypertarget{eq-black-scholes}{}{
\frac{\partial \mathrm C}{ \partial \mathrm t } + \frac{1}{2}\sigma^{2} \mathrm S^{2}
\frac{\partial^{2} \mathrm C}{\partial \mathrm C^2}
  + \mathrm r \mathrm S \frac{\partial \mathrm C}{\partial \mathrm S}\ =
  \mathrm r \mathrm C 
}\label{eq-black-scholes}\end{equation}

\hypertarget{other-blocks}{%
\subsection{Other Blocks}\label{other-blocks}}

\begin{longtable}[]{@{}
  >{\raggedright\arraybackslash}p{(\columnwidth - 2\tabcolsep) * \real{0.4167}}
  >{\raggedright\arraybackslash}p{(\columnwidth - 2\tabcolsep) * \real{0.3750}}@{}}
\toprule\noalign{}
\begin{minipage}[b]{\linewidth}\raggedright
Markdown Syntax
\end{minipage} & \begin{minipage}[b]{\linewidth}\raggedright
Output
\end{minipage} \\
\midrule\noalign{}
\endhead
\bottomrule\noalign{}
\endlastfoot
\begin{minipage}[t]{\linewidth}\raggedright
\begin{verbatim}
> Blockquote
\end{verbatim}
\end{minipage} & \begin{minipage}[t]{\linewidth}\raggedright
\begin{quote}
Blockquote
\end{quote}
\end{minipage} \\
\begin{minipage}[t]{\linewidth}\raggedright
\begin{verbatim}
| Line Block
|   Spaces and newlines
|   are preserved
\end{verbatim}
\end{minipage} & \begin{minipage}[t]{\linewidth}\raggedright
Line Block\\
\hspace*{0.333em}\hspace*{0.333em}\hspace*{0.333em}Spaces and newlines\\
\hspace*{0.333em}\hspace*{0.333em}\hspace*{0.333em}are preserved
\end{minipage} \\
\end{longtable}

\hypertarget{sec-crf}{%
\subsection{Cross-references}\label{sec-crf}}

\begin{figure}

{\centering \includegraphics{sunflower.png}

}

\caption{\label{fig-sunflower}A sunflower}

\end{figure}

\begin{longtable}[]{@{}
  >{\raggedright\arraybackslash}p{(\columnwidth - 2\tabcolsep) * \real{0.5405}}
  >{\raggedright\arraybackslash}p{(\columnwidth - 2\tabcolsep) * \real{0.4595}}@{}}
\toprule\noalign{}
\begin{minipage}[b]{\linewidth}\raggedright
Markdown Format
\end{minipage} & \begin{minipage}[b]{\linewidth}\raggedright
Output
\end{minipage} \\
\midrule\noalign{}
\endhead
\bottomrule\noalign{}
\endlastfoot
\begin{minipage}[t]{\linewidth}\raggedright
\begin{verbatim}
@fig-sunflower is pretty.
\end{verbatim}
\end{minipage} & Figure~\ref{fig-sunflower} is pretty. \\
\begin{minipage}[t]{\linewidth}\raggedright
\begin{verbatim}
@tbl-glm was created from code.
\end{verbatim}
\end{minipage} & Table~\ref{tbl-glm} was created from code. \\
\begin{minipage}[t]{\linewidth}\raggedright
\begin{verbatim}
@sec-crf is this section.
\end{verbatim}
\end{minipage} & Section~\ref{sec-crf} is this section. \\
\begin{minipage}[t]{\linewidth}\raggedright
\begin{verbatim}
@eq-black-scholes is above.
\end{verbatim}
\end{minipage} & Equation~\ref{eq-black-scholes} is above. \\
\end{longtable}

See the
\href{https://quarto.org/docs/authoring/cross-references.html}{Quarto
documentation on cross-references for more}.

\hypertarget{citations}{%
\section{Citations}\label{citations}}

This section of the template is adapted from the
\href{https://quarto.org/docs/authoring/footnotes-and-citations.html}{Quarto
citation documentation}.

Quarto supports bibliography files in a wide variety of formats
including BibTeX and CSL. Add a bibliography to your document using the
\texttt{bibliography} YAML metadata field. For example:

\begin{Shaded}
\begin{Highlighting}[]
\PreprocessorTok{{-}{-}{-}}
\FunctionTok{title}\KeywordTok{:}\AttributeTok{ }\StringTok{"My Document"}
\FunctionTok{bibliography}\KeywordTok{:}\AttributeTok{ references.bib}
\PreprocessorTok{{-}{-}{-}}
\end{Highlighting}
\end{Shaded}

See the \href{https://pandoc.org/MANUAL.html\#citations}{Pandoc
Citations} documentation for additional information on bibliography
formats.

\hypertarget{citations-1}{%
\section{Citations}\label{citations-1}}

This section of the template is adapted from the
\href{https://quarto.org/docs/authoring/footnotes-and-citations.html}{Quarto
citation documentation}.

Quarto supports bibliography files in a wide variety of formats
including BibTeX and CSL. Add a bibliography to your document using the
\texttt{bibliography} YAML metadata field. For example:

\begin{Shaded}
\begin{Highlighting}[]
\PreprocessorTok{{-}{-}{-}}
\FunctionTok{title}\KeywordTok{:}\AttributeTok{ }\StringTok{"My Document"}
\FunctionTok{bibliography}\KeywordTok{:}\AttributeTok{ references.bib}
\PreprocessorTok{{-}{-}{-}}
\end{Highlighting}
\end{Shaded}

See the \href{https://pandoc.org/MANUAL.html\#citations}{Pandoc
Citations} documentation for additional information on bibliography
formats.

\hypertarget{sec-citations}{%
\subsection{Citation Syntax}\label{sec-citations}}

Quarto uses the standard Pandoc markdown representation for citations.
Here are some examples:

\begin{longtable}[]{@{}
  >{\raggedright\arraybackslash}p{(\columnwidth - 2\tabcolsep) * \real{0.3860}}
  >{\raggedright\arraybackslash}p{(\columnwidth - 2\tabcolsep) * \real{0.6140}}@{}}
\toprule\noalign{}
\begin{minipage}[b]{\linewidth}\raggedright
Markdown Format
\end{minipage} & \begin{minipage}[b]{\linewidth}\raggedright
Output
\end{minipage} \\
\midrule\noalign{}
\endhead
\bottomrule\noalign{}
\endlastfoot
\begin{minipage}[t]{\linewidth}\raggedright
\begin{verbatim}
Blah Blah [see @knuth1984, pp. 33-35;
also @wickham2015, chap. 1]
\end{verbatim}
\end{minipage} & Blah Blah (see Knuth 1984, 33--35; also Wickham 2015,
chap. 1) \\
\begin{minipage}[t]{\linewidth}\raggedright
\begin{verbatim}
Blah Blah [@knuth1984, pp. 33-35,
38-39 and passim]
\end{verbatim}
\end{minipage} & Blah Blah (Knuth 1984, 33--35, 38--39 and passim) \\
\begin{minipage}[t]{\linewidth}\raggedright
\begin{verbatim}
Blah Blah [@wickham2015; @knuth1984].
\end{verbatim}
\end{minipage} & Blah Blah (Wickham 2015; Knuth 1984). \\
\begin{minipage}[t]{\linewidth}\raggedright
\begin{verbatim}
Wickham says blah [-@wickham2015]
\end{verbatim}
\end{minipage} & Wickham says blah (2015) \\
\end{longtable}

You can also write in-text citations, as follows:

\begin{longtable}[]{@{}
  >{\raggedright\arraybackslash}p{(\columnwidth - 2\tabcolsep) * \real{0.5000}}
  >{\raggedright\arraybackslash}p{(\columnwidth - 2\tabcolsep) * \real{0.4444}}@{}}
\toprule\noalign{}
\begin{minipage}[b]{\linewidth}\raggedright
Markdown Format
\end{minipage} & \begin{minipage}[b]{\linewidth}\raggedright
Output
\end{minipage} \\
\midrule\noalign{}
\endhead
\bottomrule\noalign{}
\endlastfoot
\begin{minipage}[t]{\linewidth}\raggedright
\begin{verbatim}
@knuth1984 says blah.
\end{verbatim}
\end{minipage} & Knuth (1984) says blah. \\
\begin{minipage}[t]{\linewidth}\raggedright
\begin{verbatim}
@knuth1984 [p. 33] says blah.
\end{verbatim}
\end{minipage} & Knuth (1984, 33) says blah. \\
\end{longtable}

See the \href{https://pandoc.org/MANUAL.html\#citations}{Pandoc
Citations} documentation for additional information on citation syntax.

To provide a custom citation stylesheet, provide a path to a CSL file
using the \texttt{csl} metadata field in your document, for example:

\begin{Shaded}
\begin{Highlighting}[]
\PreprocessorTok{{-}{-}{-}}
\FunctionTok{title}\KeywordTok{:}\AttributeTok{ }\StringTok{"My Document"}
\FunctionTok{bibliography}\KeywordTok{:}\AttributeTok{ references.bib}
\FunctionTok{csl}\KeywordTok{:}\AttributeTok{ nature.csl}
\PreprocessorTok{{-}{-}{-}}
\end{Highlighting}
\end{Shaded}

\newpage{}

\hypertarget{references}{%
\section*{References}\label{references}}
\addcontentsline{toc}{section}{References}

\hypertarget{refs}{}
\begin{CSLReferences}{1}{0}
\leavevmode\vadjust pre{\hypertarget{ref-knuth1984}{}}%
Knuth, Donald E. 1984. {``Literate Programming.''} \emph{The Computer
Journal} 27 (2): 97--111.

\leavevmode\vadjust pre{\hypertarget{ref-wickham2015}{}}%
Wickham, Hadley. 2015. \emph{R Packages}. 1st ed. O'Reilly Media, Inc.

\end{CSLReferences}



\end{document}
