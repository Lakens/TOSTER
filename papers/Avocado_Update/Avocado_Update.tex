% interactcadsample.tex
% v1.03 - April 2017

\documentclass[]{interact}

\usepackage{epstopdf}% To incorporate .eps illustrations using PDFLaTeX, etc.
\usepackage{subfigure}% Support for small, `sub' figures and tables
%\usepackage[nolists,tablesfirst]{endfloat}% To `separate' figures and tables from text if required

\usepackage{natbib}% Citation support using natbib.sty
\bibpunct[, ]{(}{)}{;}{a}{}{,}% Citation support using natbib.sty
\renewcommand\bibfont{\fontsize{10}{12}\selectfont}% Bibliography support using natbib.sty

\theoremstyle{plain}% Theorem-like structures provided by amsthm.sty
\newtheorem{theorem}{Theorem}[section]
\newtheorem{lemma}[theorem]{Lemma}
\newtheorem{corollary}[theorem]{Corollary}
\newtheorem{proposition}[theorem]{Proposition}

\theoremstyle{definition}
\newtheorem{definition}[theorem]{Definition}
\newtheorem{example}[theorem]{Example}

\theoremstyle{remark}
\newtheorem{remark}{Remark}
\newtheorem{notation}{Notation}

% Pandoc syntax highlighting
\usepackage{color}
\usepackage{fancyvrb}
\newcommand{\VerbBar}{|}
\newcommand{\VERB}{\Verb[commandchars=\\\{\}]}
\DefineVerbatimEnvironment{Highlighting}{Verbatim}{commandchars=\\\{\}}
% Add ',fontsize=\small' for more characters per line
\usepackage{framed}
\definecolor{shadecolor}{RGB}{248,248,248}
\newenvironment{Shaded}{\begin{snugshade}}{\end{snugshade}}
\newcommand{\AlertTok}[1]{\textcolor[rgb]{0.94,0.16,0.16}{#1}}
\newcommand{\AnnotationTok}[1]{\textcolor[rgb]{0.56,0.35,0.01}{\textbf{\textit{#1}}}}
\newcommand{\AttributeTok}[1]{\textcolor[rgb]{0.77,0.63,0.00}{#1}}
\newcommand{\BaseNTok}[1]{\textcolor[rgb]{0.00,0.00,0.81}{#1}}
\newcommand{\BuiltInTok}[1]{#1}
\newcommand{\CharTok}[1]{\textcolor[rgb]{0.31,0.60,0.02}{#1}}
\newcommand{\CommentTok}[1]{\textcolor[rgb]{0.56,0.35,0.01}{\textit{#1}}}
\newcommand{\CommentVarTok}[1]{\textcolor[rgb]{0.56,0.35,0.01}{\textbf{\textit{#1}}}}
\newcommand{\ConstantTok}[1]{\textcolor[rgb]{0.00,0.00,0.00}{#1}}
\newcommand{\ControlFlowTok}[1]{\textcolor[rgb]{0.13,0.29,0.53}{\textbf{#1}}}
\newcommand{\DataTypeTok}[1]{\textcolor[rgb]{0.13,0.29,0.53}{#1}}
\newcommand{\DecValTok}[1]{\textcolor[rgb]{0.00,0.00,0.81}{#1}}
\newcommand{\DocumentationTok}[1]{\textcolor[rgb]{0.56,0.35,0.01}{\textbf{\textit{#1}}}}
\newcommand{\ErrorTok}[1]{\textcolor[rgb]{0.64,0.00,0.00}{\textbf{#1}}}
\newcommand{\ExtensionTok}[1]{#1}
\newcommand{\FloatTok}[1]{\textcolor[rgb]{0.00,0.00,0.81}{#1}}
\newcommand{\FunctionTok}[1]{\textcolor[rgb]{0.00,0.00,0.00}{#1}}
\newcommand{\ImportTok}[1]{#1}
\newcommand{\InformationTok}[1]{\textcolor[rgb]{0.56,0.35,0.01}{\textbf{\textit{#1}}}}
\newcommand{\KeywordTok}[1]{\textcolor[rgb]{0.13,0.29,0.53}{\textbf{#1}}}
\newcommand{\NormalTok}[1]{#1}
\newcommand{\OperatorTok}[1]{\textcolor[rgb]{0.81,0.36,0.00}{\textbf{#1}}}
\newcommand{\OtherTok}[1]{\textcolor[rgb]{0.56,0.35,0.01}{#1}}
\newcommand{\PreprocessorTok}[1]{\textcolor[rgb]{0.56,0.35,0.01}{\textit{#1}}}
\newcommand{\RegionMarkerTok}[1]{#1}
\newcommand{\SpecialCharTok}[1]{\textcolor[rgb]{0.00,0.00,0.00}{#1}}
\newcommand{\SpecialStringTok}[1]{\textcolor[rgb]{0.31,0.60,0.02}{#1}}
\newcommand{\StringTok}[1]{\textcolor[rgb]{0.31,0.60,0.02}{#1}}
\newcommand{\VariableTok}[1]{\textcolor[rgb]{0.00,0.00,0.00}{#1}}
\newcommand{\VerbatimStringTok}[1]{\textcolor[rgb]{0.31,0.60,0.02}{#1}}
\newcommand{\WarningTok}[1]{\textcolor[rgb]{0.56,0.35,0.01}{\textbf{\textit{#1}}}}

% tightlist command for lists without linebreak
\providecommand{\tightlist}{%
  \setlength{\itemsep}{0pt}\setlength{\parskip}{0pt}}



\usepackage{hyperref}
\usepackage[utf8]{inputenc}
\def\tightlist{}
\hypersetup{
  colorlinks=true,
  citecolor = cyan,
  linkcolor=blue,
  filecolor=magenta,      
  urlcolor=blue
  }

\usepackage{float}

\begin{document}


\articletype{PREPRINT}

\title{Exploring Equivalence Testing with the Updated TOSTER R Package}


\author{\name{Aaron R. Caldwell$^{a}$}
\affil{$^{a}$Natick, MA, \url{https://orcid.org/0000-0002-4541-6283}}
}

\thanks{CONTACT Aaron R.
Caldwell. Email: \href{mailto:arcaldwell49@gmail.com}{\nolinkurl{arcaldwell49@gmail.com}}}

\maketitle

\begin{abstract}
Equivalence testing is arguably under utilized by experimental
researchers. Due to limited software support for such analyses, and
little education on the topic in graduate programs, the utilization of
equivalence testings still appares to be low. One option for equivalence
testing is the use of two one-sided tests (TOST). The TOSTER R package
and jamovi module, originally developed by Daniel Lakens in 2017, was
created to make TOST more accessible to the average researcher. In the
past two years, I have made significant changes to the TOSTER package in
order to increase its accessibility and provide more robust analysis
options for researchers. In this paper, I will detail the changes to the
package and highlight new analysis options that will make TOST easier
for the average quantitative researcher.
\end{abstract}

\begin{keywords}
statistics, bootstrap, minimal effects test, NHST, TOST
\end{keywords}

\hypertarget{introduction}{%
\section{Introduction}\label{introduction}}

Researchers often erroneously declare that no statistical effect exists
based on a single ``non-significant'' p-value \citep{blandaltman95}. In
many of these cases, the data may corroborate the researcher's claim,
but the interpretation of a null hypothesis significance test (NHST),
wherein the lack of significance is considered evidence of ``no
effect'', is nonetheless incorrect. In order to statistically test for
whether there is practically no effect, researchers could use
equivalence testing. Equivalence testing is used when the goal of a
statistical test is to demonstrate that the difference between two
conditions is too small to be meaningful. For example, if a researcher
wanted to test whether a new drug was no worse than a standard drug, the
null hypothesis would be that the new drug is worse than the standard
drug by more than a meaningful amount, and the alternative hypothesis
would be that the difference between the two drugs is small enough to be
meaningless. A very simple equivalence testing approach is the use of
``two one-sided tests'' (TOST) \citep{schuirmann1987}.

The TOST procedure is a statistical test of whether a parameter (e.g.,
mean difference) is within a specified interval. The TOST procedure can
be used to test the equivalence of two means, two proportions, two
regression coefficients, and even two variances. An upper (\(\Delta_U\))
and lower (\(\Delta_L\)) equivalence bound is specified based on the
smallest effect size of interest (SESOI). If the TOST is below a
pre-specified alpha level, then the effect can be considered close
enough to zero to be practically equivalent \citep{lakens_ori}.

Both the complaints about erroneous conclusions regarding equivalence
\citep{blandaltman95} and proposed statistical solutions
\citep{schuirmann1987} have existed for decades now. Yet, the problem
appears to persist in many applied disciplines. I believe the continued
dissonance is due to a general lack of education on equivalence testing
and a struggle for many applied researchers to implement equivalence
testing. In my experience, most researchers have received some degree of
statistical training in their doctoral or master's studies, but it is
rare that any have idea of how to use TOST. It may also be difficult to
implement equivalence testing for many researchers. This may be caused
by most statistical software defaulting to a null hypothesis of zero, or
even completely lacking an ability to change the null hypothesis.
Therefore, I feel the continued development of educational content on
TOST, and software to help with such analyses, would be beneficial to
many quantitative researchers.

The TOSTER R package\footnote{All updates to the package can be found on
  the package's website \url{https://aaroncaldwell.us/TOSTERpkg}} was
originally developed in by \citet{lakens_ori} to introduce experimental
psychologists to the concept of equivalence testing and provide an
easy-to-use implementation in R. In the years since that publication, I
have made a significant update to the package in order to improve the
user interface and expand the tools available within the package. An
experienced R programmer may have no problem performing equivalence
testing within R, but beginners may struggle with both writing the code
and interpreting the output. If you fall into that category, I would
suggest using jamovi, an open-source statistical software, that has a
TOSTER module to perform equivalence/TOST analyses. Not all the features
listed in this manuscript are available in the jamovi module, but it is
a good starting point for most researchers without statistical
programming experience.

In this manuscript, I will detail the updates to the TOSTER package, and
give some basic usage examples of some of the new functions. This is
meant to just be an introduction to \emph{how} to perform such analyses,
and provide a little bit of context for when such analyses are
appropriate. For a greater introduction to equivalence testing, I would
suggest reading other methodological tutorials
\citep{lakens_ori, lakens2018equivalence, lakens2020improving, mazzolari2022myths}.

\hypertarget{tost-with-t-tests}{%
\section{TOST with t-tests}\label{tost-with-t-tests}}

In an effort to make TOSTER more informative and easier to use, a new
function \texttt{t\_TOST} was created. This function operates very
similarly to base R's \texttt{t.test} function, but performs 3 t-tests
(one two-tailed and two one-tailed tests). In addition, this function
has a generic method where two vectors can be supplied or a formula can
be given (e.g.,\texttt{y\ \textasciitilde{}\ group}). This function also
makes it easier to switch between types of t-tests. All three types (two
sample, one sample, and paired samples) can be performed/calculated from
the same function. Moreover, the output from this function is verbose,
and should make the decisions derived from the function more informative
and user-friendly.

Also, \texttt{t\_TOST} is not limited to equivalence tests. Minimal
effects testing (MET) is possible. MET is useful for situations where
the hypothesis is about a minimal effect and the \emph{null hypothesis
is equivalence} (see Figure 1) \citep{mazzolari2022myths}.

\begin{figure}
\centering
\includegraphics{Avocado_Update_files/figure-latex/hypplot-1.pdf}
\caption{Type of Hypothesis}
\end{figure}

\newpage

In these examples of \texttt{t\_TOST}, we will use the \texttt{bugs}
data from the \texttt{jmv} R package and the \texttt{sleep} data.

\begin{Shaded}
\begin{Highlighting}[]
\FunctionTok{data}\NormalTok{(}\StringTok{\textquotesingle{}sleep\textquotesingle{}}\NormalTok{)}
\FunctionTok{library}\NormalTok{(jmv)}
\FunctionTok{data}\NormalTok{(}\StringTok{\textquotesingle{}bugs\textquotesingle{}}\NormalTok{)}
\end{Highlighting}
\end{Shaded}

\hypertarget{independent-groups}{%
\subsection{Independent Groups}\label{independent-groups}}

For this example, we will use the sleep data. In this data, there is a
\texttt{group} variable and an outcome \texttt{extra}.

\begin{Shaded}
\begin{Highlighting}[]
\FunctionTok{head}\NormalTok{(sleep,}\DecValTok{2}\NormalTok{)}
\end{Highlighting}
\end{Shaded}

\begin{verbatim}
##   extra group ID
## 1   0.7     1  1
## 2  -1.6     1  2
\end{verbatim}

We will assume the data are independent (in reality this is paired
data), and that we have equivalence bounds of +/- 0.5 units of
\texttt{extra}. All we need to do is provide the \texttt{formula},
\texttt{data}, and \texttt{eqb} arguments for the function to run
appropriately. In addition, we can set the \texttt{var.equal} argument
(to assume equal variance), and the \texttt{paired} argument (sets if
the data is paired or not). Both are logical indicators that can be set
to TRUE or FALSE. The \texttt{alpha} is automatically set to 0.05 but
this can also be adjusted by the user depending on the desired
alpha-level\footnote{I strongly recommend users ``justify their alpha''
  \citep{jya1, jya2}, and the justification process can be aided by my
  other R package \href{https://aaroncaldwell.us/Superpower}{Superpower}}.

Standardize mean differences (SMDs) are provided in the output for any
t-test based TOST analysis (e.g., Cohen's d). The Hedges's corrected SMD
\citep{hedges_bias} is automatically calculated, but this can be
overridden with the \texttt{bias\_correction} argument\footnote{Glass's
  delta can also be produced in the output by using the \texttt{glass}
  argument}. In previous versions of this package, the equivalence
bounds could be set by the SMD (e.g., equivalence bound of 0.5 SD), but
this is an erroneous approach since the bound would be dependent upon
the \emph{sample} variance. However, users can opt for such an analysis
by setting \texttt{eqbound\_type} to SMD, which will produce a
noticeable warning to the R console.

The \texttt{hypothesis} argument is automatically set to ``EQU'' for
equivalence, but if a minimal effect is of interest then ``MET'' can be
supplied.

\begin{Shaded}
\begin{Highlighting}[]
\CommentTok{\# Formula Interface}
\NormalTok{res1 }\OtherTok{=} \FunctionTok{t\_TOST}\NormalTok{(}\AttributeTok{formula =}\NormalTok{ extra }\SpecialCharTok{\textasciitilde{}}\NormalTok{ group, }\AttributeTok{data =}\NormalTok{ sleep, }
              \AttributeTok{eqb =}\NormalTok{ .}\DecValTok{5}\NormalTok{, }\AttributeTok{smd\_ci =} \StringTok{"t"}\NormalTok{)}
\CommentTok{\# x \& y Interface}
\NormalTok{res1a }\OtherTok{=} \FunctionTok{t\_TOST}\NormalTok{(}\AttributeTok{x =} \FunctionTok{subset}\NormalTok{(sleep,group}\SpecialCharTok{==}\DecValTok{1}\NormalTok{)}\SpecialCharTok{$}\NormalTok{extra,}
               \AttributeTok{y =} \FunctionTok{subset}\NormalTok{(sleep,group}\SpecialCharTok{==}\DecValTok{2}\NormalTok{)}\SpecialCharTok{$}\NormalTok{extra, }\AttributeTok{eqb =}\NormalTok{.}\DecValTok{5}\NormalTok{)}
\end{Highlighting}
\end{Shaded}

Once the function has run, we can print the results with the
\texttt{print} method. This provides a verbose summary of the results.

\begin{Shaded}
\begin{Highlighting}[]
\FunctionTok{print}\NormalTok{(res1)}
\end{Highlighting}
\end{Shaded}

\begin{verbatim}
## 
## Welch Two Sample t-test
## 
## The equivalence test was non-significant, t(17.78) = -1.272, p = 8.9e-01
## The null hypothesis test was non-significant, t(17.78) = -1.861, p = 7.94e-02
## NHST: don't reject null significance hypothesis that the effect is equal to zero 
## TOST: don't reject null equivalence hypothesis
## 
## TOST Results 
##                 t    df p.value
## t-test     -1.861 17.78   0.079
## TOST Lower -1.272 17.78   0.890
## TOST Upper -2.450 17.78   0.012
## 
## Effect Sizes 
##                Estimate     SE               C.I. Conf. Level
## Raw             -1.5800 0.8491 [-3.0534, -0.1066]         0.9
## Hedges's g(av)  -0.7965 0.5992  [-1.8362, 0.2433]         0.9
## Note: SMD confidence intervals are an approximation. See vignette("SMD_calcs").
\end{verbatim}

\newpage

Another nice feature is the generic \texttt{plot} method that can
provide a visual summary of the results. Most of the plots in this
package were inspired by the
\href{https://cran.r-project.org/package=concurve}{concurve} R package
\citep{rafi2020}. There are two types of plots that can be produced. The
first, and default, is the consonance density plot
(\texttt{type\ =\ "cd"}).

\begin{Shaded}
\begin{Highlighting}[]
\FunctionTok{plot}\NormalTok{(res1, }\AttributeTok{type =} \StringTok{"cd"}\NormalTok{)}
\end{Highlighting}
\end{Shaded}

\begin{figure}
\centering
\includegraphics{Avocado_Update_files/figure-latex/cdplot-1.pdf}
\caption{Example of consonance density plot.}
\end{figure}

\newpage

The shading pattern can be modified with the \texttt{ci\_shades}.

\begin{Shaded}
\begin{Highlighting}[]
\FunctionTok{plot}\NormalTok{(res1, }\AttributeTok{type =} \StringTok{"cd"}\NormalTok{,}
     \AttributeTok{ci\_shades =} \FunctionTok{c}\NormalTok{(.}\DecValTok{9}\NormalTok{,.}\DecValTok{95}\NormalTok{))}
\end{Highlighting}
\end{Shaded}

\begin{figure}
\centering
\includegraphics{Avocado_Update_files/figure-latex/shadeplot-1.pdf}
\caption{Demonstrating the shading in plot method.}
\end{figure}

\newpage

Consonance plots, where all confidence intervals can be simultaneous
plotted, can also be produced. The advantage here is multiple confidence
interval lines can plotted at once.

\begin{Shaded}
\begin{Highlighting}[]
\FunctionTok{plot}\NormalTok{(res1, }\AttributeTok{type =} \StringTok{"c"}\NormalTok{,}
     \AttributeTok{ci\_lines =}  \FunctionTok{c}\NormalTok{(.}\DecValTok{9}\NormalTok{,.}\DecValTok{95}\NormalTok{))}
\end{Highlighting}
\end{Shaded}

\begin{figure}
\centering
\includegraphics{Avocado_Update_files/figure-latex/conplot-1.pdf}
\caption{Example of consonance plot.}
\end{figure}

\newpage

\hypertarget{paired-sample}{%
\subsection{Paired Sample}\label{paired-sample}}

To perform TOST on paired samples, the process does not change much. We
could process the test the same way by providing a formula. All we would
need to then is change \texttt{paired} to TRUE.

\begin{Shaded}
\begin{Highlighting}[]
\NormalTok{res2 }\OtherTok{=} \FunctionTok{t\_TOST}\NormalTok{(}\AttributeTok{formula =}\NormalTok{ extra }\SpecialCharTok{\textasciitilde{}}\NormalTok{ group,}
              \AttributeTok{data =}\NormalTok{ sleep,}
              \AttributeTok{paired =} \ConstantTok{TRUE}\NormalTok{,}
              \AttributeTok{eqb =}\NormalTok{ .}\DecValTok{5}\NormalTok{)}
\NormalTok{res2}
\end{Highlighting}
\end{Shaded}

\begin{verbatim}
## 
## Paired t-test
## 
## The equivalence test was non-significant, t(9) = -2.777, p = 9.89e-01
## The null hypothesis test was significant, t(9) = -4.062, p = 2.83e-03
## NHST: reject null significance hypothesis that the effect is equal to zero 
## TOST: don't reject null equivalence hypothesis
## 
## TOST Results 
##                 t df p.value
## t-test     -4.062  9   0.003
## TOST Lower -2.777  9   0.989
## TOST Upper -5.348  9 < 0.001
## 
## Effect Sizes 
##               Estimate    SE               C.I. Conf. Level
## Raw             -1.580 0.389   [-2.293, -0.867]         0.9
## Hedges's g(z)   -1.174 0.411 [-1.8046, -0.4977]         0.9
## Note: SMD confidence intervals are an approximation. See vignette("SMD_calcs").
\end{verbatim}

\newpage

However, we may have two vectors of data that are paired. So instead we
may want to just provide those separately rather than using a data set
and setting the formula. This can be demonstrated with the ``bugs''
data.

\begin{Shaded}
\begin{Highlighting}[]
\NormalTok{res3 }\OtherTok{=} \FunctionTok{t\_TOST}\NormalTok{(}\AttributeTok{x =}\NormalTok{ bugs}\SpecialCharTok{$}\NormalTok{LDHF,}
              \AttributeTok{y =}\NormalTok{ bugs}\SpecialCharTok{$}\NormalTok{LDLF,}
              \AttributeTok{paired =} \ConstantTok{TRUE}\NormalTok{,}
              \AttributeTok{eqb =} \DecValTok{1}\NormalTok{)}
\NormalTok{res3}
\end{Highlighting}
\end{Shaded}

\begin{verbatim}
## 
## Paired t-test
## 
## The equivalence test was non-significant, t(90) = 2.655, p = 9.95e-01
## The null hypothesis test was significant, t(90) = 6.649, p = 2.22e-09
## NHST: reject null significance hypothesis that the effect is equal to zero 
## TOST: don't reject null equivalence hypothesis
## 
## TOST Results 
##                 t df p.value
## t-test      6.649 90 < 0.001
## TOST Lower 10.642 90 < 0.001
## TOST Upper  2.655 90   0.995
## 
## Effect Sizes 
##               Estimate     SE             C.I. Conf. Level
## Raw             1.6648 0.2504  [1.2487, 2.081]         0.9
## Hedges's g(z)   0.6911 0.1167 [0.4987, 0.8802]         0.9
## Note: SMD confidence intervals are an approximation. See vignette("SMD_calcs").
\end{verbatim}

\newpage

Additionally, a MET, instead of equivalence testing, can be performed
with the \texttt{hypothesis} argument set to ``MET''. With this setting,
the hypothesis being tested is whether the effect is \emph{greater} than
the equivalence bound.

\begin{Shaded}
\begin{Highlighting}[]
\NormalTok{res3a }\OtherTok{=} \FunctionTok{t\_TOST}\NormalTok{(}\AttributeTok{x =}\NormalTok{ bugs}\SpecialCharTok{$}\NormalTok{LDHF,}
               \AttributeTok{y =}\NormalTok{ bugs}\SpecialCharTok{$}\NormalTok{LDLF,}
               \AttributeTok{paired =} \ConstantTok{TRUE}\NormalTok{,}
               \AttributeTok{hypothesis =} \StringTok{"MET"}\NormalTok{,}
               \AttributeTok{eqb =} \DecValTok{1}\NormalTok{)}
\NormalTok{res3a}
\end{Highlighting}
\end{Shaded}

\begin{verbatim}
## 
## Paired t-test
## 
## The minimal effect test was significant, t(90) = 10.642, p = 4.69e-03
## The null hypothesis test was significant, t(90) = 6.649, p = 2.22e-09
## NHST: reject null significance hypothesis that the effect is equal to zero 
## TOST: reject null MET hypothesis
## 
## TOST Results 
##                 t df p.value
## t-test      6.649 90 < 0.001
## TOST Lower 10.642 90       1
## TOST Upper  2.655 90   0.005
## 
## Effect Sizes 
##               Estimate     SE             C.I. Conf. Level
## Raw             1.6648 0.2504  [1.2487, 2.081]         0.9
## Hedges's g(z)   0.6911 0.1167 [0.4987, 0.8802]         0.9
## Note: SMD confidence intervals are an approximation. See vignette("SMD_calcs").
\end{verbatim}

The data would indicate that we should accept the MET hypothesis.

\newpage

\hypertarget{one-sample-t-test}{%
\subsection{One Sample t-test}\label{one-sample-t-test}}

In other cases we may have a one sample test. If that is the case, only
\texttt{x} argument for the data is needed. This is useful in situations
where you may have hypotheses to test about a single samples mean. In
order for the two-sample test to be correct, we also need to supply the
\texttt{mu} argument. In the example below, we hypothesize that the mean
of \texttt{LDHF} is not more than 1.5 points greater or less than 7.
With the way the \texttt{mu} and \texttt{eqb} arguments are set, we are
testing whether the mean of \texttt{LDHF} is significantly different
from 7.5 (two-tailed tests) and (\(\pm\)) than 1.5 points 7.5 as well
(equivalence bounds at 5.5 and 8.5).

\begin{Shaded}
\begin{Highlighting}[]
\NormalTok{res4 }\OtherTok{=} \FunctionTok{t\_TOST}\NormalTok{(}\AttributeTok{x =}\NormalTok{ bugs}\SpecialCharTok{$}\NormalTok{LDHF,}
              \AttributeTok{hypothesis =} \StringTok{"EQU"}\NormalTok{,}
              \AttributeTok{mu =} \FloatTok{7.5}\NormalTok{,}
              \AttributeTok{eqb =} \FunctionTok{c}\NormalTok{(}\FloatTok{5.5}\NormalTok{,}\FloatTok{8.5}\NormalTok{))}
\NormalTok{res4}
\end{Highlighting}
\end{Shaded}

\begin{verbatim}
## 
## One Sample t-test
## 
## The equivalence test was significant, t(90) = -4.244, p = 2.66e-05
## The null hypothesis test was non-significant, t(90) = -0.458, p = 6.48e-01
## NHST: don't reject null significance hypothesis that the effect is equal to 7.5 
## TOST: reject null equivalence hypothesis
## 
## TOST Results 
##                  t df p.value
## t-test     -0.4577 90   0.648
## TOST Lower  7.1156 90 < 0.001
## TOST Upper -4.2444 90 < 0.001
## 
## Effect Sizes 
##            Estimate     SE             C.I. Conf. Level
## Raw         -0.1209 0.2641  [6.9402, 7.818]         0.9
## Hedges's g   2.9047 0.2395 [2.5058, 3.2949]         0.9
## Note: SMD confidence intervals are an approximation. See vignette("SMD_calcs").
\end{verbatim}

We would conclude that \texttt{LDHF} is practically equivalent to the
hypothesized mean (7.5).

\newpage

\hypertarget{using-summary-statistics}{%
\subsection{Using Summary Statistics}\label{using-summary-statistics}}

In some cases you may only have access to the summary statistics (e.g.,
when reviewing an article or attempting to perform a meta-analysis).
Therefore, I created a function, \texttt{tsum\_TOST}, to perform the
same tests just based on the summary statistics. This involves providing
the function with a number of different arguments.

\begin{itemize}
\tightlist
\item
  \texttt{n1\ \&\ n2} the sample sizes (only n1 needs to be provided for
  one sample case)
\item
  \texttt{m1\ \&\ m2} the sample means
\item
  \texttt{sd1\ \&\ sd2} the sample standard deviation
\item
  \texttt{r12} the correlation between each if paired is set to
  TRUE\footnote{The \texttt{extract\_r\_paired} function can be used if
    the correlation between paired observations is not readily
    available.}
\end{itemize}

The results from the \texttt{bugs} example can be replicated with the
\texttt{tsum\_TOST}:

\begin{Shaded}
\begin{Highlighting}[]
\NormalTok{res\_tsum }\OtherTok{=} \FunctionTok{tsum\_TOST}\NormalTok{(}
  \AttributeTok{m1 =} \FunctionTok{mean}\NormalTok{(bugs}\SpecialCharTok{$}\NormalTok{LDHF, }\AttributeTok{na.rm=}\ConstantTok{TRUE}\NormalTok{), }\AttributeTok{sd1 =} \FunctionTok{sd}\NormalTok{(bugs}\SpecialCharTok{$}\NormalTok{LDHF, }\AttributeTok{na.rm=}\ConstantTok{TRUE}\NormalTok{),}
  \AttributeTok{n1 =} \FunctionTok{length}\NormalTok{(}\FunctionTok{na.omit}\NormalTok{(bugs}\SpecialCharTok{$}\NormalTok{LDHF)),}
  \AttributeTok{hypothesis =} \StringTok{"EQU"}\NormalTok{, }\AttributeTok{smd\_ci =} \StringTok{"t"}\NormalTok{, }\AttributeTok{eqb =} \FunctionTok{c}\NormalTok{(}\FloatTok{5.5}\NormalTok{, }\FloatTok{8.5}\NormalTok{)}
\NormalTok{)}

\NormalTok{res\_tsum}
\end{Highlighting}
\end{Shaded}

\begin{verbatim}
## 
## One-sample t-Test
## 
## The equivalence test was significant, t(90) = -4.244, p = 2.66e-05
## The null hypothesis test was significant, t(90) = 27.942, p = 3.91e-46
## NHST: reject null significance hypothesis that the effect is equal to zero 
## TOST: reject null equivalence hypothesis
## 
## TOST Results 
##                 t df p.value
## t-test     27.942 90 < 0.001
## TOST Lower  7.116 90 < 0.001
## TOST Upper -4.244 90 < 0.001
## 
## Effect Sizes 
##            Estimate     SE             C.I. Conf. Level
## Raw           7.379 0.2641  [6.9402, 7.818]         0.9
## Hedges's g    2.905 0.2395 [2.4289, 3.3804]         0.9
## Note: SMD confidence intervals are an approximation. See vignette("SMD_calcs").
\end{verbatim}

\newpage

\hypertarget{robust-methods-for-equivalence-testing}{%
\section{Robust Methods for Equivalence
Testing}\label{robust-methods-for-equivalence-testing}}

In some cases, the use of t-test may be less than ideal. Any serious
violation to the assumptions of a t-test (e.g., normality or
homoscedasticity) could greatly inflate the type 1 error rate of TOST.
Therefore, it may be useful to explore alternatives to the t-test for
TOST that either do not have those assumptions or are robust to
violating those assumptions.

The TOSTER package currently provides 4 robust alternatives to the
t-test for TOST. First, there is the \texttt{wilcox\_TOST} function
which uses the Wilcoxon-Mann-Whitney (WMW) type tests (i.e.,
\texttt{wilcox.test}) to perform TOST as a test of symmetry. Second,
there is the \texttt{boot\_t\_TOST} function which uses the bootstrap
method outlined by \citet{efron93}. Third, there is the
\texttt{log\_TOST} function which performs log-transformed t-tests,
which is a parametric approach commonly used in pharmaceutical
bioequivalence studies on ratio data \citep{he2022}. Fourth, there is
the \texttt{boot\_log\_TOST} function which uses the same bootstrap
method outlined by \citet{efron93} but on the log-transformed data,
which is more robust than parametric log t-test \citep{he2022}.

In the following sections, I will briefly outline the available robust
TOST functions within the TOSTER package.

\hypertarget{tests-of-symmetry-rank-based-tests}{%
\subsection{Tests of Symmetry (rank based
tests)}\label{tests-of-symmetry-rank-based-tests}}

The WMW group of tests (e.g., Mann-Whitney U-test) provide a
non-parametric test of differences between groups, or within samples,
based on \emph{ranks}. This provides a test of location shift, which is
a fancy way of saying differences in the center of the distribution
(i.e., in parametric tests the location is mean). Within the TOST
framework, there are two separate tests of directional location shift to
determine if the location shift is within (equivalence) or outside
(minimal effect) the equivalence bounds. Many researchers mistakenly
think these are tests of medians, but this is not the case (See
\citet{median_test} for details). Using a WMW-based TOST is useful for
testing whether the differences between groups/conditions is symmetric
around the equivalence bounds\footnote{Care should be taken when
  considering paired samples; a test on the rank transformed data
  \citep{kornbrot1990rank} or another robust test may be more prudent.}.
For equivalence testing, the TOST would be testing whether there is
asymmetry towards no effect with a null hypothesis of symmetry at the
equivalence bound.

In the TOSTER package, we accomplish this ``test of symmetry'' with the
\texttt{wilcox\_TOST} function. This function operates in an extremely
similar implementation to the \texttt{t\_TOST} function. The exact
calculations utilized in this function can be explored via the
documentation of the \texttt{wilcox.test} function. A standardized mean
difference (SMD) is \emph{not} calculated in this function since this
would be an inappropriate measure of effect size alongside the
non-parametric test statistics. Instead, a standardized effect size
(SES) is calculated for \emph{all} types of comparisons (e.g., two
sample, one sample, and paired samples). The function can produce a
rank-biserial correlation \citep{Kerby_2014}, a WMW Odds
\citep{wmwodds}, or a ``common language effect size'' \citep{Kerby_2014}
(Also known as the non-parametric probability of superiority, or
concordance probability).\footnote{There is no plotting capability at
  this time for the output of this function.}

\newpage

As an example, we can use the sleep data to make a non-parametric
comparison of equivalence.

\begin{Shaded}
\begin{Highlighting}[]
\NormalTok{test1 }\OtherTok{=} \FunctionTok{wilcox\_TOST}\NormalTok{(}\AttributeTok{formula =}\NormalTok{ extra }\SpecialCharTok{\textasciitilde{}}\NormalTok{ group,}
                      \AttributeTok{data =}\NormalTok{ sleep,}
                      \AttributeTok{paired =} \ConstantTok{FALSE}\NormalTok{,}
                      \AttributeTok{eqb =}\NormalTok{ .}\DecValTok{5}\NormalTok{)}
\FunctionTok{print}\NormalTok{(test1)}
\end{Highlighting}
\end{Shaded}

\begin{verbatim}
## 
## Wilcoxon rank sum test with continuity correction
## 
## The equivalence test was non-significant W = 20.000, p = 8.94e-01
## The null hypothesis test was non-significant W = 25.500, p = 6.93e-02
## NHST: don't reject null significance hypothesis that the effect is equal to zero 
## TOST: don't reject null equivalence hypothesis
## 
## TOST Results 
##            Test Statistic p.value
## NHST                 25.5   0.069
## TOST Lower           34.0   0.894
## TOST Upper           20.0   0.013
## 
## Effect Sizes 
##                           Estimate               C.I. Conf. Level
## Median of Differences       -1.346       [-3.4, -0.1]         0.9
## Rank-Biserial Correlation   -0.490 [-0.7493, -0.1005]         0.9
\end{verbatim}

Based on these results, we would have conclude there is no significant
difference but not equivalent differences either (i.e., inconclusive
result).

\newpage

\hypertarget{bootstrap-tost}{%
\subsection{Bootstrap TOST}\label{bootstrap-tost}}

The bootstrap refers to resampling with replacement and can be used for
statistical estimation and inference. Bootstrapping techniques are very
useful because they are considered somewhat robust to the violations of
assumptions for a simple t-test and provide better estimations of SMDs
\citep{Kirby2013}. Therefore, I added a bootstrapping function,
\texttt{boot\_t\_TOST}, to the package to provide another robust
alternative to the \texttt{t\_TOST} function.

In this function we provide a percentile bootstrap solution outlined by
\citet{efron93} (see chapter 16, page 220). The bootstrapped p-values
are derived from the ``studentized'' version of a test of mean
differences \citep{efron93}. Overall, the results should be similar to
the results of \texttt{t\_TOST}. \textbf{However}, for paired samples,
the Cohen's d(rm) effect size \emph{cannot} be calculated by this
function.

\hypertarget{two-sample-algorithm}{%
\subsubsection{Two Sample Algorithm}\label{two-sample-algorithm}}

The steps by which the bootstrapping occurs are fairly simple.

\begin{enumerate}
\def\labelenumi{\arabic{enumi}.}
\item
  Form B bootstrap data sets from x* and y* wherein x* is sampled with
  replacement from \(\tilde x_1,\tilde x_2, ... \tilde x_n\) and y* is
  sampled with replacement from
  \(\tilde y_1,\tilde y_2, ... \tilde y_n\)
\item
  t is then evaluated on each sample, but the mean of each sample (y or
  x) and the overall average (z) are subtracted from each (i.e., null
  distribution is formed)
\end{enumerate}

\[
t(z^{*b}) = \frac {(\bar x^*-\bar x - \bar z) - (\bar y^*-\bar y - \bar z)}{\sqrt {sd_y^*/n_y + sd_x^*/n_x}}
\]

\begin{enumerate}
\def\labelenumi{\arabic{enumi}.}
\setcounter{enumi}{2}
\tightlist
\item
  An approximate p-value can then be calculated as the number of
  bootstrapped results greater than the observed t-statistic from the
  sample.
\end{enumerate}

\[
p_{boot} = \frac {\#t(z^{*b}) \ge t_{sample}}{B}
\]

The same process is completed for the one sample case but with the one
sample solution for the equation outlined by \(t(z^{*b})\). The paired
sample case in this bootstrap procedure is equivalent to the one sample
solution because the test is based on the difference scores.

\newpage

\hypertarget{example-of-bootsrapping}{%
\subsubsection{Example of Bootsrapping}\label{example-of-bootsrapping}}

We can use the sleep data to see an example of the bootstrapped results.
If you plot the bootstrap samples, it will show how the resampling via
bootstrapping indicates the instability of Hedges' d(z). Just looking at
the printed results you will notice some differences between confidence
intervals from the bootstrapped result and the t-test.

\begin{Shaded}
\begin{Highlighting}[]
\FunctionTok{set.seed}\NormalTok{(}\DecValTok{891111}\NormalTok{)}
\NormalTok{test1 }\OtherTok{=} \FunctionTok{boot\_t\_TOST}\NormalTok{(}\AttributeTok{formula =}\NormalTok{ extra }\SpecialCharTok{\textasciitilde{}}\NormalTok{ group,}
                    \AttributeTok{data =}\NormalTok{ sleep,}
                    \AttributeTok{paired =} \ConstantTok{TRUE}\NormalTok{,}
                    \AttributeTok{eqb =}\NormalTok{ .}\DecValTok{5}\NormalTok{,}
                    \AttributeTok{R =} \DecValTok{999}\NormalTok{)}


\FunctionTok{print}\NormalTok{(test1)}
\end{Highlighting}
\end{Shaded}

\begin{verbatim}
## 
## Bootstrapped Paired t-test
## 
## The equivalence test was non-significant, t(9) = -2.777, p = 1e+00
## The null hypothesis test was significant, t(9) = -4.062, p = 0e+00
## NHST: reject null significance hypothesis that the effect is equal to zero 
## TOST: don't reject null equivalence hypothesis
## 
## TOST Results 
##                 t df p.value
## t-test     -4.062  9 < 0.001
## TOST Lower -2.777  9       1
## TOST Upper -5.348  9 < 0.001
## 
## Effect Sizes 
##               Estimate     SE               C.I. Conf. Level
## Raw             -1.580 0.3699    [-2.26, -1.038]         0.9
## Hedges's g(z)   -1.174 0.6491 [-2.7507, -0.9285]         0.9
## Note: percentile bootstrap method utilized.
\end{verbatim}

\newpage

\hypertarget{log-tost}{%
\subsection{Log TOST}\label{log-tost}}

The natural logarithmic (log) transformation is often utilized to
stabilize the variance of a measure, and it often provides the best
approximation of the normal distribution \citep{logtest}. However,
another, less often reported, advantage of the log transformation is
that the back transformation of the differences of the log-transformed
data is a \emph{ratio} \citep{logtest}. For example, if we had a two
samples (x \& y) with an geometric mean\footnote{The mean of
  log-transformed data is the \emph{geometric} not \emph{arithmetic}
  mean. I highly recommend reading \citet{logtest} and
  \citet{caldwell2019basic} for more details} or 7 and 10.5, x and y
respectively in the code below, we could represent the differences as
ratio of y:x where y is 1.5 times greater than x.

\begin{Shaded}
\begin{Highlighting}[]
\NormalTok{x }\OtherTok{=} \DecValTok{7}\NormalTok{; y }\OtherTok{=} \FloatTok{10.5}
\FunctionTok{log}\NormalTok{(y) }\SpecialCharTok{{-}} \FunctionTok{log}\NormalTok{(x)}
\end{Highlighting}
\end{Shaded}

\begin{verbatim}
## [1] 0.4054651
\end{verbatim}

\begin{Shaded}
\begin{Highlighting}[]
\FunctionTok{log}\NormalTok{(y}\SpecialCharTok{/}\NormalTok{x)}
\end{Highlighting}
\end{Shaded}

\begin{verbatim}
## [1] 0.4054651
\end{verbatim}

\begin{Shaded}
\begin{Highlighting}[]
\FunctionTok{exp}\NormalTok{(}\FunctionTok{log}\NormalTok{(y) }\SpecialCharTok{{-}} \FunctionTok{log}\NormalTok{(x))}
\end{Highlighting}
\end{Shaded}

\begin{verbatim}
## [1] 1.5
\end{verbatim}

\begin{Shaded}
\begin{Highlighting}[]
\NormalTok{y}\SpecialCharTok{/}\NormalTok{x}
\end{Highlighting}
\end{Shaded}

\begin{verbatim}
## [1] 1.5
\end{verbatim}

The log transformation thereby acts as a useful tool help tame data into
conforming to the normality assumption, and makes the interpretation
fairly simple. In addition, some regulatory agencies, such as the United
States Food and Drug Administration (FDA) \citep{fda}, specifically
require bioequivalence studies to report the geometric means and make
statistical comparisons on the log transformed data \citep{he2022}. In
pharmaceutical reserach, bioequivalence testing involves determining
whether two drugs, a test drug and a reference drug, have the same rate
and extent of absorption in the body. This is typically accomplished by
testing whether the blood concentrations of the drug after
administration of the test drug are sufficiently close to the blood
concentrations after administration of the reference drug. If the two
drugs are bioequivalent, they can be used interchangeably. The area
under the curve (AUC) is the measure of the extent of absorption, and
the peak concentration is the measure of the rate of absorption. In
order to determine bioequivalence, the AUC and peak concentration of the
test drug must be within a certain percentage of the AUC and peak
concentration of the reference drug.

In my personal experience as a physiologist, it is not uncommon that
biological/physiological phenomenon present have longer right-tailed
distributions, and are often adequately normalized with a natural log
transformation. The additional advantage is the how equivalence bounds
can, almost, be universally applied when making comparisons on the log
scale. The FDA considers to drugs to be bioequivalent when the maximal
concentration and AUC differences between drugs are less than 1.25. To
put it another way, ratio between two means must be between 1.25 and 0.8
(i.e., 1/1.25) \citep{fda}.

Therefore, I have implemented two functions to allow for the comparison
of data that is believed to be left skewed (long right tail), and is on
a ratio scale\footnote{Ratio scale means the outcome is measured on a
  numerical scale that has equal distances between adjacent values and
  true zero.}. The first function is a parametric t-test on the log
transformed scale while the second function is a bootstrapping test
which is more robust than parametric version \citep{he2022}.

\hypertarget{example-of-log-tost}{%
\subsubsection{Example of Log TOST}\label{example-of-log-tost}}

The \texttt{log\_TOST} function is almost exactly the same as the
\texttt{t\_TOST} function. First, the primary differences is that it
only accepts paired and two sample comparisons. One sample tests are not
support (i.e., there is no ratio to calculate). Second, standardized
mean differences are not calculated, but a ratio of means is instead
reported \citep{lajeunesse2015bias}\footnote{Also, referred to as a
  ``response ratio'' in ecology. Like an SMD, the response ratio can be
  utilized in meta-analysis.}. Third, the default equivalence bounds are
by default set to the FDA standards (i.e., \texttt{eqb\ =\ 1.25}), but
can be changed by the user\footnote{Only one value needs to be supplied
  to eqb; the reciprocal value of eqb is taken as the other equivalence
  bound. For example, if \texttt{eqb\ \ =\ 0.85} then the upper
  equivalence bound is 1/0.85 (\textasciitilde1.333)}.

As an example we can use the \texttt{mtcars} data to compare the type of
transmission (\texttt{am}) effects on the gas mileage (\texttt{mpg}). We
can see from the data below there are significant, non-equivalent,
differences in mpg between transmission types.

\begin{Shaded}
\begin{Highlighting}[]
\FunctionTok{log\_TOST}\NormalTok{(mpg }\SpecialCharTok{\textasciitilde{}}\NormalTok{ am, }\AttributeTok{data =}\NormalTok{ mtcars)}
\end{Highlighting}
\end{Shaded}

\begin{verbatim}
## 
## Log-transformed Welch Two Sample t-test
## 
## The equivalence test was non-significant, t(23.96) = -1.363, p = 9.07e-01
## The null hypothesis test was significant, t(23.96) = -3.826, p = 8.19e-04
## NHST: reject null significance hypothesis that the effect is equal to one 
## TOST: don't reject null equivalence hypothesis
## 
## TOST Results 
##                 t    df p.value
## t-test     -3.826 23.96 < 0.001
## TOST Lower -1.363 23.96   0.907
## TOST Upper -6.288 23.96 < 0.001
## 
## Effect Sizes 
##                  Estimate      SE               C.I. Conf. Level
## log(Means Ratio)  -0.3466 0.09061 [-0.5017, -0.1916]         0.9
## Means Ratio        0.7071      NA   [0.6055, 0.8256]         0.9
\end{verbatim}

\newpage

\hypertarget{example-of-bootstrap-log-tost}{%
\subsubsection{Example of Bootstrap Log
TOST}\label{example-of-bootstrap-log-tost}}

The bootstrap version of \texttt{log\_TOST}, \texttt{boot\_log\_TOST},
uses the same bootstrapping method detailed above
(\texttt{boot\_t\_TOST}), but it uses the log-transformed values and
produces the ratio of means as the effect size.

\begin{Shaded}
\begin{Highlighting}[]
\FunctionTok{boot\_log\_TOST}\NormalTok{(mpg }\SpecialCharTok{\textasciitilde{}}\NormalTok{ am, }\AttributeTok{data =}\NormalTok{ mtcars, }\AttributeTok{R=}\DecValTok{999}\NormalTok{)}
\end{Highlighting}
\end{Shaded}

\begin{verbatim}
## 
## Bootstrapped Log Welch Two Sample t-test
## 
## The equivalence test was non-significant, t(23.96) = -1.363, p = 9.57e-01
## The null hypothesis test was significant, t(23.96) = -3.826, p = 0e+00
## NHST: reject null significance hypothesis that the effect is equal to 1 
## TOST: don't reject null equivalence hypothesis
## 
## TOST Results 
##                 t    df p.value
## t-test     -3.826 23.96 < 0.001
## TOST Lower -1.363 23.96   0.957
## TOST Upper -6.288 23.96 < 0.001
## 
## Effect Sizes 
##                  Estimate      SE               C.I. Conf. Level
## log(Means Ratio)  -0.3466 0.08634 [-0.4871, -0.2039]         0.9
## Means Ratio        0.7071 0.06119   [0.6144, 0.8156]         0.9
## Note: percentile bootstrap method utilized.
\end{verbatim}

From this analysis, we would conclude there is a significant effect that
is not practically equivalent.

\newpage

\hypertarget{equivalence-testing-with-anovas}{%
\section{Equivalence Testing with
ANOVAs}\label{equivalence-testing-with-anovas}}

Many researchers utilize ANOVA as an omnibus test for the
absence/presence of effects before inspecting multiple pairwise
comparisons. This is very useful when implementing factorial designs
wherein multiple experimental factors are tested and/or manipulated. As
\citet{Campbell_2021} suggest, the lack of a significant result at the
ANOVA-level does not necessarily indicate that a factor or interaction
of factors have no effect. However, \citet{Campbell_2021} only suggest
an equivalence test for one-way ANOVAs and therefore exclude
multi-factor or factorial ANOVAs. Therefore, I have extended the work of
\citet{Campbell_2021} to include functions that allow for equivalence
testing of the partial \(\eta^2\) (eta-squared) effect size from ANOVAs.

\hypertarget{f-test-calculations}{%
\subsection{F-test Calculations}\label{f-test-calculations}}

Statistical equivalence testing\footnote{Also called ``omnibus
  non-inferiority testing'' by \citet{Campbell_2021}} for \emph{F}-tests
are special use case of the cumulative distribution function of the
non-central \emph{F} distribution. As \citet{Campbell_2021} states, this
type of statistical test answers the question: ``Can we reject the
hypothesis that the total proportion of variance in outcome Y
attributable to X is greater than or equal to the equivalence bound
\(\Delta\)?''

\hypertarget{hypothesis-tests}{%
\subsubsection{Hypothesis Tests}\label{hypothesis-tests}}

\[
H_0 =  1 > \eta^2_p \geq \Delta
\]

\[
H_1 =  0 \geq \eta^2_p < \Delta
\]

In TOSTER, I have gone a tad farther than \citet{Campbell_2021}, and
have included a calculation for a generalization of the non-centrality
parameter that allows the equivalence test for \emph{F}-tests to be
applied to variety of designs.

\citet{Campbell_2021} calculate the \emph{p}-value as:

\[
p = p_f(F; J-1, N-J, \frac{N \cdot \Delta}{1-\Delta})
\]

The non-centrality parameter (ncp = \(\lambda\)) can be calculated with
the equivalence bound and the degrees of freedom:

\[
\lambda_{eq} = \frac{\Delta}{1-\Delta} \cdot(df_1 + df_2 +1)
\]

\newpage

The \emph{p}-value for the equivalence test (\(p_{eq}\)) could then be
calculated from traditional ANOVA results and the distribution function:

\[
p_{eq} = p_f(F; df_1, df_2, \lambda_{eq})
\]

\hypertarget{example-of-equivalence-anova-testing}{%
\subsection{Example of Equivalence ANOVA
Testing}\label{example-of-equivalence-anova-testing}}

Using the \texttt{InsectSprays} data set in R and the base R
\texttt{aov} function, I can demonstrate how this omnibus equivalence
testing can be applied with TOSTER. From the initial analysis we an see
a clear ``significant'' effect (very small p-value) of the inspect
spray. However, we \emph{may} be interested in testing if the effect is
practically equivalent. I will arbitrarily set the equivalence bound to
a partial eta-squared of 0.35 (\(H_0: \eta^2_p > 0.35\)).

\begin{Shaded}
\begin{Highlighting}[]
\FunctionTok{data}\NormalTok{(}\StringTok{"InsectSprays"}\NormalTok{)}
\NormalTok{aovtest }\OtherTok{=} \FunctionTok{aov}\NormalTok{(count }\SpecialCharTok{\textasciitilde{}}\NormalTok{ spray, }\AttributeTok{data =}\NormalTok{ InsectSprays)}
\FunctionTok{anova}\NormalTok{(aovtest)}
\end{Highlighting}
\end{Shaded}

\begin{verbatim}
## Analysis of Variance Table
## 
## Response: count
##           Df Sum Sq Mean Sq F value    Pr(>F)    
## spray      5 2668.8  533.77  34.702 < 2.2e-16 ***
## Residuals 66 1015.2   15.38                      
## ---
## Signif. codes:  0 '***' 0.001 '**' 0.01 '*' 0.05 '.' 0.1 ' ' 1
\end{verbatim}

We can then use the information in the table above to perform an
equivalence test using the \texttt{equ\_ftest} function. This function
returns an object of the S3 class \texttt{htest} and the output will
look very familiar to that of the t-test. The main difference is the
estimates, and confidence interval, are for partial \(\eta^2_p\).

\begin{Shaded}
\begin{Highlighting}[]
\FunctionTok{equ\_ftest}\NormalTok{(}\AttributeTok{Fstat =} \FloatTok{34.70228}\NormalTok{,  }\AttributeTok{df1 =} \DecValTok{5}\NormalTok{, }\AttributeTok{df2 =} \DecValTok{66}\NormalTok{,  }\AttributeTok{eqb =} \FloatTok{0.35}\NormalTok{)}
\end{Highlighting}
\end{Shaded}

\begin{verbatim}
## 
##  Equivalence Test from F-test
## 
## data:  Summary Statistics
## F = 34.702, df1 = 5, df2 = 66, p-value = 1
## 95 percent confidence interval:
##  0.5806263 0.7804439
## sample estimates:
## [1] 0.724439
\end{verbatim}

Based on the results above we would conclude there is a significant
effect of ``spray'' and the differences due to spray are \emph{not}
statistically equivalent. In essence, we reject the traditional null
hypothesis of ``no effect'' but accept the null hypothesis of the
equivalence test.

\newpage

The \texttt{equ\_ftest} function is very useful because all you need is
very basic summary statistics. However, if you are doing all your
analyses in R then you can use the \texttt{equ\_anova} function. This
function accepts objects produced from \texttt{stats::aov},
\texttt{car::Anova} and \texttt{afex::aov\_car} (or any ANOVA from
derived from \texttt{afex}).

As a second example, we can use the afex package's data and ANOVA
\citep{afex}. Again, we will use the equivalence bound of 0.35, which is
a completely arbitrary (and baseless) equivalence bound. Notice that the
output contains 2 p-values: one for the significance (\texttt{p.null})
and another for the equivalence test (\texttt{p.equ}).

\begin{Shaded}
\begin{Highlighting}[]
\CommentTok{\# Example using a purely within{-}subjects design }
\CommentTok{\# (Maxwell \& Delaney, 2004, Chapter 12, Table 12.5, p. 578):}
\FunctionTok{library}\NormalTok{(afex)}
\FunctionTok{data}\NormalTok{(md\_12}\FloatTok{.1}\NormalTok{)}
\NormalTok{aovtest2 }\OtherTok{=} \FunctionTok{aov\_ez}\NormalTok{(}\StringTok{"id"}\NormalTok{, }\StringTok{"rt"}\NormalTok{, md\_12}\FloatTok{.1}\NormalTok{, }\AttributeTok{within =} \FunctionTok{c}\NormalTok{(}\StringTok{"angle"}\NormalTok{, }\StringTok{"noise"}\NormalTok{), }
       \AttributeTok{anova\_table=}\FunctionTok{list}\NormalTok{(}\AttributeTok{correction =} \StringTok{"none"}\NormalTok{, }\AttributeTok{es =} \StringTok{"none"}\NormalTok{))}
\FunctionTok{equ\_anova}\NormalTok{(aovtest2,}
          \AttributeTok{eqb =} \FloatTok{0.35}\NormalTok{)}
\end{Highlighting}
\end{Shaded}

\begin{verbatim}
##        effect df1 df2   F.value       p.null       pes eqbound     p.equ
## 1 (Intercept)   1   9 598.44917 1.526600e-09 0.9851839    0.35 0.9999997
## 2       angle   2  18  40.71910 2.086763e-07 0.8189831    0.35 0.9992557
## 3       noise   1   9  33.76596 2.559737e-04 0.7895522    0.35 0.9763228
## 4 angle:noise   2  18  45.31034 9.424093e-08 0.8342857    0.35 0.9996103
\end{verbatim}

\newpage

\hypertarget{equivalence-between-replication-studies}{%
\section{Equivalence Between Replication
Studies}\label{equivalence-between-replication-studies}}

During the development of this TOSTER update, I was helping advise a
team of researchers on a massive replication project for sport and
exercise science \citep{repSES}. How to determine whether a
direct\footnote{Defined as being a as-close-as possible replication to
  the original study, in contrast to ``conceptual'' replications.}
replication was a successful replication of the original study was
contentious topic of conversation among the team. Inspired by these
discussions, I created 2 functions that would utilize the basic
principles of SMDs\footnote{The textbook by \citet{borenstein} and the
  some of the works of Wolfgang Vietchbauer, metafor R package author,
  were a large source of information for developing these functions.} to
test for differences between two studies.

Overall, the concept is simple: if we have estimates of SMDs from two
very similar studies we can use the large-sample approximation to
compute the sampling variances\footnote{Users can also supply their own
  sampling variances using the \texttt{se1} and \texttt{se2} arguments.}
to estimate the degree to which the two studies differ from one another
(i.e., calculate p-values). The users of TOSTER then have the option to
test whether the two SMDs significantly differ, or use TOST to estimate
if they are practically equivalent. Additionally, there are two options
for comparing SMDs: using the summary statistics or using bootstrapping
(assuming original data is available).

\hypertarget{example-using-summary-statistics}{%
\subsection{Example using Summary
Statistics}\label{example-using-summary-statistics}}

In this example, let us imagine an ``original'' study that reports an
effect of Cohen's dz = 0.95 in a paired samples design with 25 subjects.
However, a replication doubled the sample size, found a non-significant
effect at an SMD of 0.2. Are these two studies compatible (the lower the
p-value the lower the compatibility)? Or, to put it another way, should
the replication be considered a ``failure'' to replicate the original
study?

We can use the \texttt{compare\_smd} function to at least measure how
often we would expect a discrepancy between the original and replication
study if the same underlying effect was being measured (also assuming no
publication bias).

We can see from the results below that, if the null hypothesis were
true, we would only expect to see a discrepancy in SMDs between studies
at least this large \textasciitilde1\% of the time.

\begin{Shaded}
\begin{Highlighting}[]
\FunctionTok{compare\_smd}\NormalTok{(}\AttributeTok{smd1 =} \FloatTok{0.95}\NormalTok{,}
            \AttributeTok{n1 =} \DecValTok{25}\NormalTok{,}
            \AttributeTok{smd2 =} \FloatTok{0.23}\NormalTok{,}
            \AttributeTok{n2 =} \DecValTok{50}\NormalTok{,}
            \AttributeTok{paired =} \ConstantTok{TRUE}\NormalTok{)}
\end{Highlighting}
\end{Shaded}

\begin{verbatim}
## 
##  Difference in Cohen's dz (paired)
## 
## data:  Summary Statistics
## z = 2.5685, p-value = 0.01021
## alternative hypothesis: true difference in SMDs is not equal to 0
## sample estimates:
## difference in SMDs 
##               0.72
\end{verbatim}

Let us also imagine a scenario where a replication team considers a
replication successful if the SMDs are within 0.25 units of each other.
We can set the \texttt{TOST} argument to TRUE, and then set the
equivalence bound using \texttt{null} argument.

\begin{Shaded}
\begin{Highlighting}[]
\FunctionTok{compare\_smd}\NormalTok{(}\AttributeTok{smd1 =} \FloatTok{0.95}\NormalTok{, }\AttributeTok{n1 =} \DecValTok{25}\NormalTok{, }\AttributeTok{smd2 =} \FloatTok{0.23}\NormalTok{,}\AttributeTok{n2 =} \DecValTok{50}\NormalTok{,}
            \AttributeTok{paired =} \ConstantTok{TRUE}\NormalTok{, }\AttributeTok{TOST =} \ConstantTok{TRUE}\NormalTok{, }\AttributeTok{null =}\NormalTok{ .}\DecValTok{25}\NormalTok{)}
\end{Highlighting}
\end{Shaded}

\begin{verbatim}
## 
##  Difference in Cohen's dz (paired)
## 
## data:  Summary Statistics
## z = 1.6767, p-value = 0.9532
## alternative hypothesis: true difference in SMDs is less than 0.25
## sample estimates:
## difference in SMDs 
##               0.72
\end{verbatim}

Based on the imaginary studies we outlined above, we would not reject
the null equivalence hypothesis, but reject the null significance
hypothesis. Therefore, we would could conclude that there are
significant differences between the studies that are not practically
equivalent.

\hypertarget{example-using-bootstrapping}{%
\subsection{Example using
Bootstrapping}\label{example-using-bootstrapping}}

The above results are only based on an approximating the differences
between the SMDs. If the raw data is available, then the optimal
solution is the bootstrap. This can be accomplished with the
\texttt{boot\_compare\_smd} function. The only drawback to this function
is that TOST is currently not avaiable, and users would instead have to
run 2 one-sided tests manually using the \texttt{null} and
\texttt{alternative} arguments.

For this example, we will simulate some data. As an alternative approach
to TOST, we can just set the \texttt{alpha} to 0.1, and then check to
see if the confidence interval is within the preset equivalence bounds.

\begin{Shaded}
\begin{Highlighting}[]
\FunctionTok{set.seed}\NormalTok{(}\DecValTok{4522}\NormalTok{)}
\NormalTok{boot\_test }\OtherTok{=} \FunctionTok{boot\_compare\_smd}\NormalTok{(}\AttributeTok{x1 =} \FunctionTok{rnorm}\NormalTok{(}\DecValTok{25}\NormalTok{,.}\DecValTok{95}\NormalTok{), }\AttributeTok{x2 =} \FunctionTok{rnorm}\NormalTok{(}\DecValTok{50}\NormalTok{), }
                             \AttributeTok{paired =} \ConstantTok{TRUE}\NormalTok{, }\AttributeTok{alpha =}\NormalTok{ .}\DecValTok{1}\NormalTok{)}
\NormalTok{boot\_test}
\end{Highlighting}
\end{Shaded}

\begin{verbatim}
## 
##  Bootstrapped Differences in SMDs (paired)
## 
## data:  Bootstrapped
## z (observed) = 2.887, p-value = 0.006003
## alternative hypothesis: true difference in SMDs is not equal to 0
## 90 percent confidence interval:
##  0.4070761 1.3508435
## sample estimates:
## difference in SMDs 
##          0.8058872
\end{verbatim}

\newpage

\hypertarget{conclusions}{%
\section{Conclusions}\label{conclusions}}

In this manuscript I have demonstrated most of the new functions and
features within the TOSTER R package. This constitutes a major update to
the package over the past 2 years. I hope that updates to the package
builds upon the original impact of the TOSTER package\footnote{In my
  opinion, the impact of the \citet{lakens_ori} cannot be overstated
  considering it is cited by over 1000 other papers!}, and has been made
TOST more accessible to the average researcher. In addition, I have
added a number of other functions that offer robust alternatives to the
t-test for performing TOST analyses. I would strongly recommend users of
TOSTER to explore these functions, and, at the very least, compare the
robust results to the t-test results to ensure that the conclusions do
not change due to the chosen analysis\footnote{If they do change, then
  it would be prudent to explore what features in the data might explain
  this discrepancy.}. Lastly, to my knowledge, this is the first package
to offer equivalence testing options for ANOVAs or for comparing SMDs
between studies. Overall, this package and its functions offer an easily
accessible option for researchers to explore equivalence testing, and
hopefully improve their statistical analyses.

\newpage

\hypertarget{additional-information}{%
\section{Additional Information}\label{additional-information}}

All analyses/code in this manuscript are from TOSTER v0.6.0:

\begin{verbatim}
# Install the exact release with this code
devtools::install_github("Lakens/TOSTER@v0.6.0")
\end{verbatim}

\hypertarget{acknowledgements}{%
\subsection*{Acknowledgement(s)}\label{acknowledgements}}
\addcontentsline{toc}{subsection}{Acknowledgement(s)}

I'd would like to thank everyone from the Lakens' laboratory group for
their input and suggestions.

\hypertarget{disclosure-statement}{%
\subsection*{Disclosure statement}\label{disclosure-statement}}
\addcontentsline{toc}{subsection}{Disclosure statement}

The author of this manuscript is the author of the TOSTER package.
Citations of this manuscript will benefit his citation count.

\hypertarget{funding}{%
\subsection*{Funding}\label{funding}}
\addcontentsline{toc}{subsection}{Funding}

No funding was provided for this work.

\hypertarget{notes-on-contributors}{%
\subsection*{Notes on contributor(s)}\label{notes-on-contributors}}
\addcontentsline{toc}{subsection}{Notes on contributor(s)}

Daniel Lakens provided a review of many of the materials that have been
incorporated into the update of TOSTER, and was the original author of
this package. Without his help and encouragment, the TOSTER package and
this update would not exist.

\hypertarget{nomenclaturenotation}{%
\subsection*{Nomenclature/Notation}\label{nomenclaturenotation}}
\addcontentsline{toc}{subsection}{Nomenclature/Notation}

\begin{itemize}
\tightlist
\item
  ANOVA: Analysis of Variance
\item
  Bootstrapping: the use of random sampling with replacement to estimate
  statistics
\item
  FDA: Food and Drug Administration (United States of America)
\item
  MET: Minimal Effects Test
\item
  ncp: non-centrality parameter
\item
  SESOI: Smallest Effect Size of Interest
\item
  SMD: Standardized Mean Difference (e.g., Cohen's d)
\item
  TOST: Two-One Sided Tests
\item
  WMW: Wilcoxon-Mann-Whitney
\end{itemize}

\hypertarget{notes}{%
\subsection*{Notes}\label{notes}}
\addcontentsline{toc}{subsection}{Notes}

The R package is also (partially) implemented in jamovi as the TOSTER
module.

\newpage

\bibliographystyle{tfcad}
\bibliography{interactcadsample.bib}





\end{document}
